\documentclass[12pt, a4paper]{report}

\usepackage{url}
\usepackage[utf8]{inputenc}
\usepackage{graphicx}
\usepackage{hyperref}

\usepackage[
    backend=biber,
    style=alphabetic,
    sorting=ynt
]{biblatex}
\addbibresource{darwinsquest.bib}

\graphicspath{{img/}} % configuring the graphicx package globally

\title{
    \begin{figure}[h]
    \centering{}
    \includegraphics[width=\textwidth]{logo} % remember best practice without extension, NO img/logo, img/logo.png or logo.png, but logo is enough
    \end{figure}
}
\author{
    Enrico Marchionni\\
    \texttt{enrico.marchionni@studio.unibo.it}
    \and
    Francesco Cipollone\\
    \texttt{francesco.cipollone@studio.unibo.it}
    \and
    Raffaele Marrazzo\\
    \texttt{raffaele.marrazzo@studio.unibo.it}
}
\date{\today}

\begin{document}

\maketitle

\begin{abstract}

    DarwinsQuest \cite{ontheoriginofspiecies} is a multi-level structured videogame set in a post apocalyptic world,
    shaken by climate change. The objective is to survive natural selection by completing
    numerous battles, after choosing your genetically modified banions\footnote{Banion = basic companion}.

\end{abstract}

\tableofcontents

\chapter{Analysis}

\section{Requirements}

\subsubsection{Functional}

\begin{itemize}
    \item The game will open with a title screen, containing buttons to start/quit the game, and optionally a load button.
    \item \textit{The player can choose between three game modes: Normal, Hard, and Impossible.}
    \item At the beginning of the game, the player must be able to select a starter fight companion from a number of multiple choices.
    \item The game interface will consist of a side-scrolling 2D view, containing the player and enemies. Encountering one enemy will prompt a battle.
    \item \textit{The game has to manage an economic system that manages win and loss currency transactions.}
    \item The player's movement will be managed by tiles. Each tile can be a move, battle, or special tile.
        The movement will be determined by the use of a random value. Said value corresponds to the number of tiles that the player can move.
    \item The companions are able to evolve based on an evolution tree. Starting from a node, the player can only select one of its children's nodes.
        Once the player has unlocked one of the leaf nodes, the last tree node, the evolution in the said tree is completed.
        \textit{Multiple trees can be present at once, meaning that multiple evolution branches could be developed at once.}
    \item The player has to sequentially complete a series of levels, with increasing difficulty until meeting the final boss.
        The battles are engaged in 1v1 a turn-based combat fashion, with the ability to switch between companions or perform moves.
        Later in the game, 2v2 battles can be triggered. After winning a battle the player can choose an upgrade from the evolution tree.
    \item The player can save their progress during the game. The player can save their progress only in the board.
        It will not be possible to save during fights.
\end{itemize}

\subsubsection{Non Functional}

\begin{itemize}
    \item The battle needs to be challenging for the player and will require a careful strategic plan planning ahead. \label{challengingbattle}
    \item The game performances must be acceptable.
    \item The game has to be portable and compatible with Windows, MacOS, and Linux systems.
\end{itemize}

\section{Domain Model}

    The player can move inside the board map bounds. The map contains the player, the enemies, and special tiles such as the shop, bonus, and penalty tiles.
    There are some different kind of tiles:
\begin{itemize}
    \item \textbf{Move tile}: a generic tile that does not contain any battle or special characteristics;
    \item \textbf{Battle tile}: landing on this tile will prompt the beginning of a battle (battle tiles cannot be skipped, regardless of the drawn value).
        Losing said battle will make the player stuck on this tile, preventing them from proceeding further in the game until they beat the opponent;
    \item \textbf{Special tile}:
    \begin{itemize}
        \item \textbf{Bonus tiles}:
        \begin{itemize}
            \item Money gain;
            \item Casual companion stats increase;
        \end{itemize}
    \end{itemize}
    \begin{itemize}
        \item \textbf{Penalty tiles}:
        \begin{itemize}
            \item Money loss;
            \item Casual companion stats decrease;
        \end{itemize}
    \end{itemize}
\end{itemize}

    The player can own one or more fight companions. Each companion has an intrinsic elemental type, such as fire, water, grass, rock, air, electro,
    and is associated has a set of moves, which are divided into groups based on the companions' elemental type. A certain companion with a certain
    type only retains moves of its same elemental type plus neutral moves, e.g. A fire companion can only use fire moves and neutral moves,
    and cannot perform other types' moves.

    Banions in our game are the following:
\begin{center}
    \begin{tabular}{| c | c | c | c | c | c | c |}
        \hline
        Banion & Element \\ [0.5ex] % Fire & Water & Grass & Rock & Air & Electro
        \hline\hline
        fatbird & Air \\
        \hline
        bluebird & Air \\
        \hline
        bee & Air \\
        \hline
        bat & Electro \\
        \hline
        bunny & Electro \\
        \hline
        chameleon & Electro \\
        \hline
        angrypig & Fire \\
        \hline
        chicken & Fire \\
        \hline
        mushroom & Fire \\
        \hline
        trunk & Grass \\
        \hline
        radish & Grass \\
        \hline
        plant & Grass \\
        \hline
        rocks & Rock \\
        \hline
        turtle & Rock \\
        \hline
        rino & Rock \\
        \hline
        slime & Water \\
        \hline
        snail & Water \\
        \hline
        duck & Water \\
        \hline
    \end{tabular}
\end{center}

    Elemental reactions, that are bounded to moves, are defined as follows:
\begin{center}
    \begin{tabular}{| c || c | c | c | c | c | c |}
        \hline
        Player \textbackslash Enemy & Fire & Water & Grass & Rock & Air & Electro \\ [0.5ex]
        \hline\hline
        Air & * & + & - & + & - & * \\
        \hline
        Electro & * & * & - & * & * & + \\
        \hline
        Fire & * & - & + & * & * & * \\
        \hline
        Grass & - & * & * & - & + & + \\
        \hline
        Rock & * & * & + & * & - & * \\
        \hline
        Water & + & * & * & * & - & * \\
        \hline
    \end{tabular}
\end{center}

    \textit{Each companion in the game will have a personal set of statistics, such as attack (ATK), defense (DEF), health points (HP).
    The values make up the base statistics of a certain companion.}

    In every level, there are two player entities: the player and the opponent (NPC). These entities deploy their companions, which will fight 1v1.
    A level consists of a selection phase, a battle phase, and leveling phase:
\begin{itemize}
    \item \textbf{Selection phase}: before battling the player has to choose the first companion in his battle lineup from his current team;
    \item \textbf{Battle phase}: during the player's fight turn, it is possible to switch their active companion to a different one (the team must have more than a single companion), 
        or the player can perform a move. Upon the player's active companion's defeat, they will be forced to switch to another one. 
        If the player is out of eligible companions the fight is over. In 2v2 battles, the player will deploy two of their companions. 
        It will be possible to select which of the two enemy companions to attack while respecting each other turns.
    \item \textbf{Leveling phase}: upon the player's victory, you will be prompted to choose an upgrade from the evolution tree, starting from any of the lastly unlocked nodes.
\end{itemize}

    The fight mechanics are considered a project challenge for their complexity \ref{challengingbattle}, from the elemental reaction logic to the moves management.

    The game currency can be obtained by winning battles and from bonus tiles (see ahead).
    Losing a battle or landing on a Money loss tile will decrease the player's money amount.
    Money can be used to purchase items, such as potions to temporarily increase your companion statistics, and companions from the in-game shop.

    Modelling an evolution tree is considered a challenge because it requires designing a complex branching system that accurately reflects the evolution of the companions in the game. 
    The tree must be visually appealing and easy to navigate, while also ensuring that each node is balanced and offers meaningful choices to the player. 
    The designer must also consider how to balance the evolution of multiple trees simultaneously and ensure that the player does not become overwhelmed with too many choices. 
    Additionally, the evolution tree must be designed to fit seamlessly into the game's mechanics and narrative.

    % => mvc.png is a fictional picture inside img folder
    % \begin{figure}[h]
    % \centering{}
    % \includegraphics[width=\textwidth]{mvc} % remember best practice without extension, NO img/mvc, img/mvc.png or mvc.png
    % \caption{Mvc pattern example.}
    % \label{img:mvc}
    % \end{figure}

\chapter{Design}

\section{Architecture}

    This projects is developed in \href{https://en.wikipedia.org/wiki/Model%E2%80%93view%E2%80%93controller}{MVC} (Model, View and Controller) architecture.

\section{Details}

    \dots

    \subsubsection{Enrico Marchionni}

    \dots

    \subsubsection{Francesco Cipollone}

    \dots

    \subsubsection{Raffaele Marrazzo}

    \dots

\chapter{Deployment}

    \dots

\section{Automatized testing}

    \dots

    \subsubsection{Enrico Marchionni}

    \dots

    \subsubsection{Francesco Cipollone}

    \dots

    \subsubsection{Raffaele Marrazzo}

    \dots

\section{Work strategy}

    \dots

\section{Development notes}

    \dots

    \subsubsection{Enrico Marchionni}

    \dots

    \subsubsection{Francesco Cipollone}

    \dots

    \subsubsection{Raffaele Marrazzo}

    \dots

\subsection{Example}

    \dots

\chapter{Final comments}

    \dots

\section{Self-evaluation and future improvements}

    \dots

    \subsubsection{Enrico Marchionni}

    \dots

    \subsubsection{Francesco Cipollone}

    \dots

    \subsubsection{Raffaele Marrazzo}

    \dots

\section{Difficulties and comments to teachers}

    \dots

    \subsubsection{Enrico Marchionni}

    \dots

    \subsubsection{Francesco Cipollone}

    \dots

    \subsubsection{Raffaele Marrazzo}

    \dots

\appendix

\chapter{User guide}

    \dots

\chapter{Laboratory}

\section{enrico.marchionni@studio.unibo.it}

\begin{itemize}
    \item Lab 04: \url{https://virtuale.unibo.it/mod/forum/discuss.php?d=113869#p169173}
    \item Lab 05: \url{https://virtuale.unibo.it/mod/forum/discuss.php?d=114647#p169723}
    \item Lab 06: \url{https://virtuale.unibo.it/mod/forum/discuss.php?d=115548#p171159}
    \item Lab 07: \url{https://virtuale.unibo.it/mod/forum/discuss.php?d=117044#p173058}
    \item Lab 08: \url{https://virtuale.unibo.it/mod/forum/discuss.php?d=117852#p174127}
    \item Lab 09: \url{https://virtuale.unibo.it/mod/forum/discuss.php?d=118995#p175326}
    \item Lab 10: \url{https://virtuale.unibo.it/mod/forum/discuss.php?d=119938#p176522}
    \item Lab 11: \url{https://virtuale.unibo.it/mod/forum/discuss.php?d=121130#p177368}
    \item Lab 12: \url{https://virtuale.unibo.it/mod/forum/discuss.php?d=121885#p178425}
\end{itemize}

\printbibliography

\end{document}